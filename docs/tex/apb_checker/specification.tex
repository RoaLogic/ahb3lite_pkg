\chapter{Specifications} \label{specifications}

\section{Functional Description}\label{functional-description}

The Roa Logic APB Checker VIP is a configurable, fully parameterized Verification IP (VIP)
that continuously and autonomosly observes and verifies all APB transactions.
The APB Checker VIP is fully compliant with the AMBA APB2, APB3, APB4, and APB5 protocols.


\section{Operating Modes}\label{operating-modes}

The APB Checker VIP supports the APB2, APB3, APB4, and APB5 bus protocols. The protocol to verify is selected using a define statement;

\texttt{`define APB\_VERSION\_APB5}

\texttt{`define APB\_VERSION\_APB4}

\texttt{`define APB\_VERSION\_APB3}

The default APB2 protocol is used when no define is set. When APB\_VERSION\_APB5 is defined, then APB\_VERSION\_APB4 is automatically defined. When APB\_VERSION\_APB4 is defined, then APB\_VERSION\_APB3 is automatically defined.
The module ports and executed rules reflect the selected protocol.
 

\subsection{PCLK}\label{PCLK}
APB is a synchronous protocol. All transactions take place on the rising edge of \texttt{PCLK}.
Most of the rules are triggered on the rising edge of \texttt{PCLK}. This has the advantage of simple rule design and fast execution. The protocol checker has a minimal simulation performance effect. The disadvantage is that the checker does not look at values inbetween clock edges. It is assumed that all APB signals, except for \texttt{PRESETn} and \texttt{PCLK}, are driven by registers or at least behave like being driven by registers.
