\chapter{Extending the VIP}\label{extending}

\section{Introduction}

The Roa Logic APB Checker VIP can be easiliy extended with custom rules. This chapter explains the structure of the VIP and how to modify it.

\section{Structure}

The VIP consits of 3 sections; the message structure, the checks and rules, and the module body.


\subsection{The Message Structure}

When a rule triggers, a message and severity is reported.
The reporting is done by the \texttt{message} task which only takes a message number as input.

\lstset{language=verilog,
        basicstyle=\small\ttfamily,
        showstringspaces=false,
        frame=single}
        
\begin{lstlisting}
task automatic message (input int msg_no);
  ...
endtask : message
\end{lstlisting}

\noindent
The message and its severity level are stored in a struct.

\begin{lstlisting}
typedef struct {
  severity_t severity;
  string     message;
} message_t;
\end{lstlisting}

\noindent
The message is stored as a SystemVerilog string, whereas the severity level is a user defined integer type.

\begin{lstlisting}
typedef enum int {OFF    =0,
                  INFO   =1,
                  WARNING=2,
                  ERROR  =3,
                  FATAL  =4} severity_t;
\end{lstlisting}

\noindent
Extend the enum to add a new severity level, like this:

\begin{lstlisting}
typedef enum int {OFF    =0,
                  INFO   =1,
                  WARNING=2,
                  ERROR  =3,
                  FATAL  =4,
                  MY_LVL =5} severity_t;
\end{lstlisting}

\pagebreak
\noindent
All messages are stored in the unpacked array \texttt{\_msg} of type \texttt{message\_t}.
The messages are loaded into the array using an initial block. This approach makes adding new messages straighforward. First increase \texttt{MESSAGE\_COUNT}, then add the message with its default severity level.

\begin{lstlisting}
initial
begin
  ...
  _msg[MESSAGE_COUNT-1] = '{MY_LVL, "My message"};
  ...
end
\end{lstlisting}



\subsection{Creating new checks/rules}

All rules and checks are written as verilog tasks. Each task has the following structure;

\begin{lstlisting}
task check_myrules
  //rule 1
  if (condition)
    message(messageNumber);
    
  //rule 2
  if (condition2)
    message(messageNumber2);
endtask : check_myrules
\end{lstlisting}

\noindent
If a check or rule is only applicable to a specific APB version, then the rule/check must be enwrapped as shown below for the \texttt{check\_pbuser} task.

\begin{lstlisting}
  /*
   * Check PBUSER
   */
`ifdef APB_VERSION_APB5
  task check_pbuser;
    //PBUSER undefined when transfer completes?
    if (PENABLE && PREADY)
      if (PBUSER === 1'bx || PBUSER === 1'bz)
        message(35);
  endtask : check_pbuser
`endif
\end{lstlisting}



\subsection{Adding the check in the main body}

The final step is adding the new check to the main body. Because APB is a synchronous protocol, almost all checks are triggered by \texttt{PCLK}.

\begin{lstlisting}
  /*
   * Check MyRules
   */
   always @(posedge PCLK) check_myrules();
\end{lstlisting}

