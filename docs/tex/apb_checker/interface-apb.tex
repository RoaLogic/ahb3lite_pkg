\section{APB Interface}\label{apb-interface}

The APB Interface is a configurable APB Interface. All signals
defined in the protocol are supported as described below. See the
\emph{AMBA APB Protocol Specifications} for a complete description
of the signals.

\begin{longtable}[]{@{}lccll@{}}
\toprule
\textbf{Port} & \textbf{Size} & \textbf{Direction} & \textbf{Version} & \textbf{Description}\tabularnewline
\midrule
\endhead
\texttt{PRESETn} & 1 & Input & APB2 & Reset\tabularnewline
\texttt{PCLK} & 1 & Input & APB2 & Clock\tabularnewline
\texttt{PSEL} & 1 & Input & APB2 & Select\tabularnewline
\texttt{PENABLE} & 1 & Input & APB2 & Enable\tabularnewline
\texttt{PADDR} & \texttt{ADDR\_WIDTH} & Input & APB2 & Address\tabularnewline
\texttt{PWRITE} & 1 & Input & APB2 & Direction\tabularnewline
\texttt{PSTRB} & \texttt{DATA\_WIDTH/8} & Input & APB4 & Write Strobe\tabularnewline
\texttt{PPROT} & 3 & Input & APB4 & Protection Type\tabularnewline
\texttt{PWDATA} & \texttt{DATA\_WIDTH} & Input & APB2 & Write Data\tabularnewline
\texttt{PRDATA} & \texttt{DATA\_WIDTH} & Input & APB2 & Read Data\tabularnewline
\texttt{PREADY} & 1 & Input & APB3 & Ready\tabularnewline
\texttt{PSLVERR} & 1 & Input & APB3 & Transfer Error\tabularnewline
\texttt{PWAKEUP} & 1 & Input & APB5 & Wake-up\tabularnewline
\texttt{PAUSER} & \texttt{USER\_REQ\_WIDTH} & Input & APB5 & User request attribute\tabularnewline
\texttt{PWUSER} & \texttt{USER\_DATA\_WIDTH} & Input & APB5 & User write data attribute\tabularnewline
\texttt{PRUSER} & \texttt{USER\_DATA\_WIDTH} & Input & APB5 & User read data attribute\tabularnewline
\texttt{PBUSER} & \texttt{USER\_RESP\_WIDTH} & Input & APB5 & User response attribute\tabularnewline
\bottomrule
\caption{APB Interface Ports}
\end{longtable}

Signals for an APB version higher than selected are not present on the interface. See the \hyperref[core-configuration]{Core Configuration} section.


\subsection{PRESETn}\label{presetn}

When the active low asynchronous \texttt{PRESETn} input is asserted (`0'), the
APB interface is put into its initial reset state. 

\subsection{PCLK}\label{pclk}

\texttt{PCLK} is the APB interface clock. All APB signals are timed against the rising edge of \texttt{PCLK}.

The APB Bus Checker VIP requires a valid \texttt{PCLK}. All checks and rules trigger on the rising edge of \texttt{PCLK}.

\subsection{PSEL}\label{psel}

The APB \emph{Requester} generates \texttt{PSEL}, signaling to a \emph{Completer} that it is selected and that a data transfer is required.

\subsection{PENABLE}\label{penable}

\texttt{PENABLE} indicates the second and subsequent cycles of a transfer. The cycles when \texttt{PENABLE} is asserted (`1') are called the \emph{Access Phase}. It is driven by the \emph{Requester}.

\subsection{PADDR}\label{paddr}

\texttt{PADDR} is the APB address bus. The bus width is defined by the \texttt{ADDR\_WIDTH} parameter.

\subsection{PWRITE}\label{pwrite}

\texttt{PWRITE} indicates the direction of the transfer. When \texttt{PWRITE} is asserted (`1') it indicates a write access and a read data access when de-asserted (`0'). It is driven by the \emph{Requester}.

\subsection{PSTRB}\label{pstrb}

\texttt{PSTRB} is an optional APB4 signal driven by the \emph{Requester}.. It indicates which byte lane to update during a
write transfer. There is one \texttt{PSTRB} signal per byte lane of the APB write data bus
(\texttt{PWDATA}), such that \texttt{PSTRB[n]} corresponds to \texttt{PWDATA[(8n+7):8n]}.

\subsection{PPROT}\label{pprot}

\texttt{PPROT} is an optional APB4 signal driven by the \emph{Requester}. It indicates the normal, privileged, or secure protection level of the transaction and whether the transaction is a data or an instruction access. \texttt{PPROT} has a width of 3 bits.

\subsection{PWDATA}\label{pwdata}

\texttt{PWDATA} is the APB write data bus and is driven by the \emph{Requester} during write cycles, when \texttt{PWRITE} is asserted (`1'). The bus width is defined by the \texttt{DATA\_WIDTH} parameter.

\subsection{PRDATA}\label{prdata}

\texttt{PRDATA} is the APB read data bus and is driven by the \emph{Completer} during read cycles, when \texttt{PWRITE} is de-asserted (`0'). The bus width is defined by the \texttt{DATA\_WIDTH} parameter.

\subsection{PREADY}\label{pready}

\texttt{PREADY} is an APB3 signal driven by the \emph{Completer}. It is used to extend an APB transfer.

\subsection{PSLVERR}\label{pslverr}

\texttt{PSLVERR} is an optional APB3 signal driven by the \emph{Completer}. It indicates an error condition on the APB bus when asserted (`1').

\subsection{PWAKEUP}\label{pwakeup}

\texttt{PWAKEUP} is an optional APB5 signal driven by the \emph{Requester}. It indicates any activity associated with an APB interface.

\subsection{PAUSER}\label{pauser}

\texttt{PAUSER} is an optional APB5 signal driven by the \emph{Requester}. The bus width is defined by the \texttt{USER\_REQ\_WIDTH} parameter.

\subsection{PWUSER}\label{pwuser}

\texttt{PWUSER} is an optional APB5 signal driven by the \emph{Requester}. The bus width is defined by the \texttt{USER\_DATA\_WIDTH} parameter.

\subsection{PRUSER}\label{pruser}

\texttt{PRUSER} is an optional APB5 signal driven by the \emph{Requester}. The bus width is defined by the \texttt{USER\_DATA\_WIDTH} parameter.

\subsection{PBUSER}\label{pbuser}

\texttt{PBUSER} is an optional APB5 signal driven by the \emph{Requester}. The bus width is defined by the \texttt{USER\_RESP\_WIDTH} parameter.